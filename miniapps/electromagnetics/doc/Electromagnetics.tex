\documentclass{article}
\usepackage{amsmath}
\usepackage{amssymb}
\usepackage{amsfonts}
\usepackage{pgf}

%=============================================================================
%                              Useful Commands
%=============================================================================
 
\newcommand{\refEq}[1]{(\ref{eq:#1})}
\newcommand{\refChap}[1]{Chapter~\ref{chap:#1}}
\newcommand{\refFig}[1]{Figure~\ref{fig:#1}}
\newcommand{\refSec}[1]{Section~\ref{sec:#1}}
\newcommand{\refTab}[1]{Table~\ref{tab:#1}}
\newcommand{\refApp}[1]{Appendix~\ref{app:#1}}
\newcommand{\newEq}[2]{\begin{equation} \label{eq:#1} #2 \end{equation}}
\newcommand{\cross}{\!\times\!}
\newcommand{\inner}{\!\cdot\!}
\newcommand{\Div}{\nabla\!\cdot\!}
\newcommand{\Divx}{\nabla_{\vec{x}}\!\cdot\!}
\newcommand{\Curl}{\nabla\!\times\!}
\newcommand{\Curlx}{\nabla_{\vec{x}}\!\times\!}
\newcommand{\Grad}{\nabla\!}
\newcommand{\Gradu}{\nabla_{\vec{u}}\!}
\newcommand{\Gradx}{\nabla_{\vec{x}}\!}
\newcommand{\st}{\Bigl\lvert\;}
\newcommand{\St}{\biggl\lvert\;}
%\newcommand{\Div}{\nabla\cdot}
%\newcommand{\Divx}{\nabla_{\vec{x}}\cdot}
%\newcommand{\Curl}{\nabla\times}
%\newcommand{\Curlx}{\nabla_{\vec{x}}\times}
%\newcommand{\Grad}{\nabla}
%\newcommand{\Gradu}{\nabla_{\vec{u}}}
%\newcommand{\Gradx}{\nabla_{\vec{x}}}

\providecommand{\abs}[1]{{\left\lvert#1\right\rvert}}
\providecommand{\norm}[1]{{\left\lVert#1\right\rVert}}

\def\Hdiv{$H(Div)$ }
\def\Hcurl{$H(Curl)$ }

%\pgfdeclareimage[width=5in]{pen_fun}{images/pen_fun_and_deriv}

%=============================================================================

\title{MFEM Electromagnetics Mini Applications}
\author{The MFEM Team}

\begin{document}

\maketitle

\section{Electromagnetics}

The equations describing electromagnetic phenomena are know
collectively as the ``Maxwell Equations''.  They are usually given as:
\begin{eqnarray}
\Curl\vec{H} - \frac{\partial\vec{D}}{\partial t} &=& \vec{J} \label{eq:ampere} \\
\Curl\vec{E} + \frac{\partial\vec{B}}{\partial t} &=& 0 \label{eq:faraday} \\
\Div\vec{D} &=& \rho \label{eq:gauss} \\
\Div\vec{B} &=& 0 \label{eq:divb}
\end{eqnarray}
Where equation~\refEq{ampere} can be referred to as {\em Amp\`ere's Law},
equation~\refEq{faraday} is called {\em Faraday's Law},
equation~\refEq{gauss} is {\em Gauss's Law}, and equation~\refEq{divb}
doesn't generally have a name but is related to the nonexistence of
magnetic monopoles.  The various fields in these equations are:
\begin{center}
\begin{tabular}{|l|l|l|}
\hline
Symbol & Name & SI Units \\
\hline
$\vec{H}$ & Magnetic Field & Ampere/meter \\
$\vec{B}$ & Magnetic Flux Density & Tesla \\
$\vec{E}$ & Electric Field & Volts/meter \\
$\vec{D}$ & Electric Displacement & Coulomb/meter$^2$ \\
$\vec{J}$ & Current Density & Ampere/meter$^2$ \\
$\rho$ & Charge Density & Coulomb/meter$^3$ \\
\hline
\end{tabular}
\end{center}
In the literature these names do vary, particularly those for
$\vec{H}$ and $\vec{B}$, but in this document we will try to adhere to
the convention laid out above.

Generally we also need constitutive relations between $\vec{E}$ and
$\vec{D}$ and/or between $\vec{H}$ and $\vec{B}$.  These relations
start with the definitions:
\begin{eqnarray}
\vec{D} &=& \epsilon_0\vec{E} + \vec{P}\\
\vec{B} &=& \mu_0\left(\vec{H} + \vec{M}\right)
\end{eqnarray}
Where $\vec{P}$ is the {\em polarization density}, and $\vec{M}$ is
the {\em magnetization}.  Also, $\epsilon_0$ is the {\em permittivity
  of free space} and $\mu_0$ is the {\em permeability of free space}
which are both constants of nature.  In many common materials the
polarization density can be approximated as a scalar multiple of the
electric field i.e. $\vec{P}=\epsilon_0\chi\vec{E}$, where $\chi$ is
called the {\em electric suscepctibility}.  In such cases we usually
use the relation $\vec{D}=\epsilon\vec{E}$ with
$\epsilon=\epsilon_0(1+\chi)$ and call $\epsilon$ the {\em
  permittivity} of the material.

The nature of magnetization is more complicated but we will take a
very simplified view which is valid in many situations.  Specifically,
we will assume that either $\vec{M}$ is proportional to $\vec{H}$
yielding the relation $\vec{B}=\mu\vec{H}$ where $\mu=\mu_0(1+\chi_M)$
and $\chi_M$ is the {\em magnetic susceptibility} or that $\vec{M}$ is
independent of the applied field.  The former case pertains to both
diamagnetic and paramagnetic materials and the later to ferromagnetic
materials.

Finally we should note that equations~\refEq{ampere} and \refEq{gauss}
can be combined to yield the equation of charge continuity
\[\frac{\partial\rho}{\partial t}+\Div\vec{J} = 0\]
which can be important in plasma physics and magnetohydrodynamics (MHD).

\subsection{Static Fields}
\subsubsection{Electrostatics}

Electrostatic problems come in a variety of subtypes but they all
derive from Gauss's Law and Faraday's Law (equations~\refEq{gauss} and
\refEq{faraday}).  When we assume no time variation, Faraday's Law
becomes simply $\Curl\vec{E}=0$. This suggests that the electric field
can be expressed as the gradient of a scalar field which is
traditionally taken to be $-\varphi$, i.e.
\begin{equation}
\vec{E} = -\Grad\varphi \label{eq:gradphi}
\end{equation}
where $\varphi$ is called the {\em electric potential} and has units
of Volts in the SI system.  Inserting this definition into
equation~\refEq{gauss} gives:
\begin{equation}
-\Div\epsilon\Grad\varphi = \rho \label{eq:poisson}
\end{equation}
which is {\em Poisson's equation} for the electric potential.  Where,
clearly, we have assumed a linear constitutive relation between
$\vec{D}$ and $\vec{E}$.  If this relation happens to be nonlinear
then Poisson's equation would need to be replaced with a more
complicated nonlinear expression.

The solutions to equation~\refEq{poisson} are non unique because they
can be shifted by any additive constant.  This means that we must
apply a Dirichlet boundary condition at at least one point in the
problem domain in order to obtain a solution.  Typically this point
will be on the boundry but it need not be so.  Such a Dirichlet value
is equivalent to fixing the voltage (aka potential) at one or more
locations.  Additionally, this equation admits a normal derivative
boundary condition.  This means setting $\hat{n}\cdot\vec{D}$ to a
prescribed value on some portion of the boundary.  This is equivalent
to defining a surface charge density on that portion of the boundary.

\subsubsection{{\tt electrostatics} Mini Application}

The {\tt electrostatics} mini application is intended to demonstrate
how to solve standard electrostatics problems in MFEM.  Its source
terms and boundary conditions are simple but they should indicate how
more specialized sources or boundary conditions could be implemented.
Note that this application assumes the mesh coordinates are given in
meters.

\begin{description}
\item[Mini Application Features:]
\item[Permittivity:] The permittivity is assumed to be that of free
  space except for an optional sphere of dielectric material which can
  be defined by the user.  The command line option {\tt -ds} can be
  used to set the parameters for this dielectric sphere.  For example,
  to produce a sphere at the origin with a radius of 0.5 and a
  relative permittivity of 3 the user would specify:
  \begin{center}{\tt -ds '0 0 0 0.5 3'}\end{center}

\item[Charge Density:] The charge density is assumed to be zero except
  for an optional sphere of uniform charge density which can be
  defined by the user.  The command line option for this is {\tt -cs}
  which follows the same pattern as the dielectric sphere.  Note that
  the last entry is the total charge of the sphere and not its charge
  density.

\item[Dirichlet BC:] Dirichlet Boundary Conditions can either specify
  piecewise constant voltages on a collection of surfaces or they can
  specify a gradient field which approximates a uniform applied
  electric field.  In either case the user specifies the surfaces
  where the Dirichlet boundary condition should be applied using the
  {\tt -dbcs} option followed by a list of boundary attributes.
  For example to select surfaces 2, 3, and 4 the user would use the
  following:
  \begin{center}{\tt -dbcs '2 3 4'}\end{center}
  To apply a gradient field ($\varphi = -z$) on these surfaces the
  user would also use the {\tt -dbcg} option.  To specify piecewise
  constant values the user would list the desired values after
  {\tt -dbcv} as follows:
  \begin{center}{\tt -dbcv '0.0 1.0 -1.0'}\end{center}

\item[Neumann BC:] Neumann Boundary Conditions set the normal
  component of the electric displacement on portions of the boundary.
  This normal component is equivalent to the surface charge density on
  the surface.  This is rarely used because surface charge densities
  are rarely known unless they are know to be zero.  However, if the
  surface charge density is zero then the Neumann BCs are not needed
  because this is the natural boundary condition.  Only piecewise
  constant Neumann boundary conditions are supported.  They can be set
  analygously to piecewise Dirichlet boundary conditions but using
  options {\tt -nbcs} and {\tt -nbcv}.

\end{description}


\subsubsection{Magnetostatics}

Magnetostatic problems arise when we assume no time variation in
Amp\`ere's Law (equation~\refEq{ampere}) which leads to:
\[\Curl\vec{H}=\vec{J}\]
In this case we'll assume a somewhat more general constitutive
relation between $\vec{H}$ and $\vec{B}$:
\[\vec{B}=\mu\vec{H}+\mu_0\vec{M} = \mu_0\left(1+\chi_M\right)\vec{H}+\mu_0\vec{M}\]
Where the magnetization is split into two portions; one which is
proportional to $\vec{H}$ and given by $\chi_M\vec{H}$, and another
which is independent of $\vec{H}$ and is given by $\vec{M}$.  This
allows for paramagnetic and/or diamagnetic materials defined through
$\mu$ as well as ferromagnetic materials represented by $\vec{M}$.
This choice yields:
\[\Curl\mu^{-1}\vec{B}=\vec{J}+\Curl\mu^{-1}\mu_0\vec{M}\]
Which, when combined with equation~\refEq{divb} becomes:
\begin{equation}
\Curl\mu^{-1}\Curl\vec{A}=\vec{J}+\Curl\mu^{-1}\mu_0\vec{M}
\end{equation}
If $\vec{J}$ happens to be zero we have another option because we can
assume that $\vec{H} = -\Grad\varphi_M$ for some scalar potential
$\varphi_M$.  When combined with equation~\refEq{divb} becomes:
\begin{equation}
\Div\mu\Grad\varphi_M = \Div\mu_0\vec{M}
\end{equation}

Once again the potential is non unique so we must apply additional
constraints in order to arrive at a solution for $\vec{A}$.  It is
common to constrain the solution by restricting the divergence of
$\vec{A}$.  In most cases this is sufficient but if the problem domain
contains handles in 3D or holes in 2D then we must also apply
a Dirichlet boundary condition at at least one degree of freedom for
each handle or hole.  This requirement is related to the fact that
each handle may be encircled by an irrotational (aka curl free) vector
field which cannot be expressed as the gradient of a scalar function.

Dirichlet boundary conditions for the vector potential on a surface
provide a means to specify the component of $\vec{B}$ normal to that
surface.  For example, setting the tangential components of $\vec{A}$
to be zero on a particular surface results in a magnetic flux density
which must be tangent to that surface.  The natural boundary condition
for the magnetic vector potential, if no Dirichlet constraints are
applied, is for the tangential components of $\Curl\vec{A}$ to be
zero.  In other words the magnetic flux density must be normal to the
surface.


more on magnetic scalar potential\ldots

\subsubsection{Statics Miniapp}

Possible sources:
\begin{itemize}
\item Charge Density $\rho$
\item Current Density $\vec{J}$
\item Magnetization $\vec{M}$
\item Surface Charge Density $\sigma$
\item Surface Current Density $\vec{K}$
\end{itemize}

\noindent Additional Boundary Conditions:
\begin{itemize}
\item Voltage at at least one point (we can set this if the user doesn't)
\item Magnetic Vector Potential on some region (restrictions???)
\item Magnetic Scalar Potential at at least one point (we may be able
  to choose this for the user)
\end{itemize}

\end{document}
